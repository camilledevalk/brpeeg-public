\begin{figure}[h]
    \centering
    \begin{subfigure}{0.5\linewidth}
        \centering
        \captionsetup{width = 0.9\linewidth}
        \includegraphics[width = \textwidth]{figures/blink-perceptrons-loss-PP.png}
        \caption{\textbf{Binary cross-entropy loss}}
        \label{fig:blink-perceptrons-loss-PP}
    \end{subfigure}%
    \begin{subfigure}{0.5\linewidth}
        \centering
        \captionsetup{width = 0.9\linewidth}
        \includegraphics[width = \textwidth]{figures/blink-perceptrons-acc-PP.png}
        \caption{\textbf{Accuracy}}
        \label{fig:blink-perceptrons-acc-PP}
    \end{subfigure}%
    \\
    \begin{subfigure}{0.5\linewidth}
        \captionsetup{width = 0.9\linewidth}
        \includegraphics[width = \textwidth]{figures/blink-perceptrons-mcc-PP.png}
        \caption{\textbf{MCC}}
        \label{fig:/blink-perceptrons-mcc-PP}
    \end{subfigure}%
    \begin{minipage}{0.5\linewidth}
        \caption{The training on the \textbf{blinks} experiment with perceptrons defined as \textbf{A} and \textbf{B} \vpageref{itemize:perceptrons}. The data was \textbf{preprocessed}; normalised and the CQ weight applied. For the test set, loss goes up which indicated overfitting.}
        \label{fig:blink-perceptrons-PP}
    \end{minipage}
\end{figure}