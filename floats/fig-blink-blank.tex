\begin{figure}[h]
    \centering
    \begin{subfigure}{0.5\textwidth}
        \centering
        \captionsetup{width = 0.9\linewidth}
        \includegraphics[width = \textwidth]{figures/signal-blink.png}
        \caption{\textbf{Blinks} -- In the data coming from the blinks, a strong signal is found just after the marking of the blink ($t=0s$).}
        \label{fig:blink}
    \end{subfigure}%
    \begin{subfigure}{0.5\textwidth}
        \centering
        \captionsetup{width = 0.9\linewidth}
        \includegraphics[width = \textwidth]{figures/signal-blank.png}
        \caption{\textbf{Blanks} -- When averaging all the examples marked as blank, there is not much signal left, only noise.}
        \label{fig:blank}
    \end{subfigure}%
    \caption{The signal coming from the \textit{blinks} recording, averaged over all the examples. The 14 different lines are produced by the 14 EEG channels. The red and grey line (channel \textsc{AF3} and \textsc{FC5}) seem really noisy.}
    \label{fig:blink-blank}
\end{figure}